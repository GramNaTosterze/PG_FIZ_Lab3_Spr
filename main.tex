\documentclass[11pt,a4paper]{article}

\usepackage[utf8]{inputenc}
\usepackage{pdfpages}
\usepackage{polski}
\usepackage{multirow}
\usepackage{gnuplottex}
\usepackage{csvsimple}
\usepackage{pgf}
\usepackage{tikz}
\usepackage{subfig}
\usepackage{float}

% !dane teraz będą trzymane w data/{nazwa}.csv i zczytywane za pomocą csvsimple!

% zmienne
\newcommand{\constM}{37.23}
\newcommand{\constG}{9.815}
\newcommand{\pgfmath}[1]{\pgfmathparse{#1}\pgfmathresult}
\newcommand{\displayData}[1]{
    \begin{tabular}{|c|c|c|c|}
        \hline U[V] & I[A] & m[g] & $I_m$[A] \\\hline
        \csvreader[
        late after line= \\\hline
        ]{#1}{1=\Im,2=\m,3=\U, 4=\Iout}
        {\U & \Im & \m & \Iout}
    \end{tabular}
}
\newcommand{\elabData}[1]{
    \begin{tabular}{|c|c|c|c|c|}
        \hline U[V] & I[A] & m[g] & $I_m$[A] & F[N] \\\hline
        \csvreader[
        late after line= \\\hline
        ]{#1}{1=\I,2=\m,3=\U, 4=\Im}
        {\U & \I & \m & \Im & \pgfmath{(\m - \constM)*\constG/1000} } % kg jest w układzie SI
    \end{tabular}
}

\begin{document}

% strona tytułowa sprawozdania
\includepdf{str_tyt.pdf}

% test używania tablety z pliku csv

\section{Wstęp}
% cel ćwiczenia: Celem ćwiczenia jest pomiar siły elektrodynamicznej (przy pomocy wagi) działającej na odcinek przewodnika z prądem, który został umieszczony w jednorodnym polu magnetycznym. Badana jest zależność tej siły od natężenia prądu płynącego w przewodniku i od indukcji pola magnetycznego. Na podstawie przeprowadzonych pomiarów wyznaczana jest wartość indukcji pola magnetycznego. Można innymi słowami po prostu cel ćwiczenia napisąć + przebieg ćwiczenia 
    W laboratorium mierzyliśmy masę przewodnika pod wpływem pola magnetycznego o różnym natężeniu. Umożliwiło nam to obliczenie siły elektrodynamicznej działającej na przewodnik. Następnie mieliśmy wyznaczyć zależność siły działającej na fragment przewodnika względem natężenia prądu płynącego w przewodniku; z otrzymanych danych wartość indukcji magnetycznej w szczelinie elektromagnesu, a na koniec zależność siły działającej na fragment przewodnika względem natężenia prądu płynącego w elektromagnesie.

\section{Wyniki laboratorium}    
    Dane otrzymane podczas pomiaru na laboratorium.\\
    \subsection{Zebrane dane}
    %Zebrane dane do zadania pierwszego
    \begin{table}[!h]
        \centering
        \displayData{Data/E5.1_5V.csv}
        \displayData{Data/E5.1_10V.csv}
        \caption{Dane do zadania pierwszego}
    \end{table}
    %Zebrane dane do zadania trzeciego
    \begin{table}[!h]
        \centering
        \begin{tabular}{|c|c|c|}
            \hline U[V] & m[g] & $I_m$[A] \\\hline
            \csvreader[
            late after line= \\\hline
            ]{Data/E5.2.csv}{1=\I,2=\m,3=\Im, 4=\U }
            {\U & \m & \Im}
        \end{tabular}
        \caption{Dane do zadania trzeciego}
    \end{table}
    
    \subsection{Niepewności przyrządów pomiarowych}
    \begin{table}[H]
        \centering
        \begin{tabular}{|c|c|c|c|}
            \hline $\Delta m_0$[g] & $\Delta m$[g] & $\Delta I[A] $ & $\Delta I_m$[A] \\\hline
            0.01 & 0.01 & 0.075 & 0.015 \\\hline
        \end{tabular}
        \caption{Przedstawienie niepewności przyrządów pomiarowych}
        \label{tab:my_label}
    \end{table}
    Wykonując pomiary do ćwiczenia E5.3 amperomierz dla I był ustawiony tak jak przy pierwszych
pomiarach, natomiast dla $I_m$ ustawiony był zakres 1A przy klasie 1,5\%.
    
    \pagebreak
\section{Opracowanie wyników}
    Rozwiązanie zadań postawionych na laboratorium
    \subsection{E5.1}
    % Treść zadanie
    \textbf{Treść zadania: } Przy stałej wartości indukcji pola magnetycznego (w naszym przypadku dla napięcia 5 V oraz 10 V) wyznaczyć zależność siły działającej na fragment przewodnika F od natężenia prądu płynącego w tym przewodniku I. Sporządzić wykres zależności F = f(I).
    \\
    Masa ramki $m_0$ = 37.23 g \\
    Siła F obliczana ze wzoru (gdzie przyjmujemy wartość g = 9.815 $\frac{m}{s^2}$): \[ F = |m - m_0| \cdot g \]

    \begin{figure}[ht!]
        \centering
        \begin{gnuplot}[terminal=epslatex]
            set title 'Graficzne przedstawienie F = f(I) dla napięcia U = 5 V
            m_0 = 37.23
            file = 'Data/E5.1_5V.csv'
            load 'Gnuplot/E5.1.gp'
        \end{gnuplot}
        \elabData{Data/E5.1_5V.csv}
        \label{fig:dane dla U = 5V}
    \end{figure}
    Tabela przedstawia zależność między siłą działającą na fragment przewodnika (F), a natężeniem prądu w przewodniku (I). Na wykresie widoczna jest również prosta uzyskana poprzez regresję liniową, która potrzebna jest do następnego zadania.
    
    \begin{figure}[ht!]
        \centering
        \begin{gnuplot}[terminal=epslatex]
            set title 'Graficzne przedstawienie F = f(I) dla napięcia U = 10 V
            m_0 = 37.23
            file = 'Data/E5.1_10V.csv'
            load 'Gnuplot/E5.1.gp'
        \end{gnuplot}
        \elabData{Data/E5.1_10V.csv}
        \label{fig:dane dla U = 10V}
    \end{figure}
    Podobnie jak w pierwszym przypadku na wykresie znajduje się prosta uzyskana poprzez regresję liniową, a tabela ukazuje oczekiwaną zależność.
    \pagebreak
    
    \subsection{E5.2}
    % Treść zadania
    \textbf{Treść zadania: } Wykorzystując zależność z zadania E5.1 wyznaczyć metodą graficzną i/lub metodą najmniejszych kwadratów wartość indukcji magnetycznej w szczelinie elektromagnesu. \\

    Korzystając z wykresu z zadania E5.1, ze wzoru $F = B I l$ oraz z informacji zawartej w uzupełnieniu, że długość fragmentu przewodnika, na który działa mierzona siła F wynosi L = 0.1 m możemy obliczyć wartość wektora indukcji B, która będzie wynosiła 
    \[ B = \frac{a}{l} =  \frac{0.001145}{0.1} = 0.01145 \: T \]  
    dla natężenia 5 V oraz 
    \[ B = \frac{a}{l} =  \frac{0.001091}{0.1} = 0.01091 \: T\]
    dla natężenia 10 V.
    \pagebreak
    
    \subsection{E5.3}
    \textbf{Treść zadania: } Przy stałym natężeniu prądu płynącym w pętli przewodnika (w naszym przypadku I = 4 A) wyznaczyć zależność siły F działającej na fragment tego przewodnika od natężenia prądu płynącego w uzwojeniu elektromagnesu Im. Określić zależność indukcji magnetycznej elektromagnesu B od Im. Sporządzić wykres zależności B = f($I_m$). \\
    
    Ponownie korzystamy ze wzoru $F = B I l$ przekształcając go aby uzyskać wartość wektora indukcji $B = \frac{F}{I\cdot l}$ 
    \begin{table}[!ht]
        \centering
        \centering
        \begin{tabular}{|c|c|c|c|c|}
            \hline U[V] & m[g] & $I_m$[A] & F[N] & B[T] \\\hline
            \csvreader[
            late after line= \\\hline
            ]{Data/E5.2.csv}{1=\I,2=\m,3=\Im, 4=\U }
            {\U & \m & \Im & \pgfmath{(\m - \constM)*\constG/1000} & \pgfmath{((\m - \constM)*\constG)/(\I * 0.1 * 1000)}}
            \multicolumn{3}{|c|}{Średnia} & 0.0376 & 0.0939\\\hline
        \end{tabular}
        \caption{Wyniki zadania E5.3}
    \end{table}
    
    Tabela przedstawia zależność między siłą działającą na fragment przewodnika (F), a natężeniem prądu w elektromagnesie ($I_m$).
    
    \begin{figure}[ht!]
        \centering
        \begin{gnuplot}[terminal=epslatex]
            set title 'Graficzne przedstawienie B = f($I_m$)
            m_0 = 37.23
            file = 'Data/E5.2.csv'
            load 'Gnuplot/E5.2.gp'
        \end{gnuplot}
        \label{fig:wykres do zadania drugiego}
    \end{figure}
    % W tym przypadku I_m jest naszym I_out
    funkcja f($I_m$) będzie mieć więc postać : 0.207728*$I_m$ + 0.001845
    \subsection{Obliczanie niepewności pomiarów}
        Niepewności złożone obliczane są za pomocą wzoru 
        \[ S_y = \sqrt{ \sum_{i = 1}^{L} (\frac{\partial y}{\partial x_1} \cdot S_{x_l} )^2 } \]

        Obliczanie niepewności F
        \[ \Delta F = \sqrt{(\frac{\partial F}{\partial m}\Delta m )^2 + (\frac{\partial F}{\partial m_0}\Delta m_0 )^2} \]
    Dla $F = |m - m_0| * g$, $\Delta m $ = 0.01 g i $\Delta m_0$ = 0.01 g otrzymujemy
        %obliczanie
        \[ \Delta F = \sqrt{ (9.815\:\frac{m}{s^2}  * 0.00001\:kg)^2 + (-9.815\:\frac{m}{s^2} * 0.00001\:kg)^2 } \]
        \[= \sqrt{(\pgfmath{9.815*0.00001}\:N)^2 + (\pgfmath{-9.815*0.00001}\:N)^2} \approx 0.00020 \:N \]

        Obliczanie niepewności B
        \[ \Delta B = \sqrt{(\frac{\partial B}{\partial F} \Delta F)^2 + (\frac{\partial B}{\partial I} \Delta I)^2} \]
        %obliczanie
        
        \[ \Delta B = \sqrt{(\frac{1}{I \cdot L} \Delta F)^2 + (-\frac{F}{I^2 \cdot L} \Delta I)^2} = \sqrt{(\frac{1}{4*0.1}* 0.000198 N)^2 + (-\frac{0.0376}{16*0.1}*0.1)^2} \]
        \[= \sqrt{(0.000495)^2 + (0.00235)^2} \approx 0.00240 \: T\]
        
    \pagebreak
\section{Wyniki} % ostateczne z niepewnościami i liczbami znaczącymi
    \subsection{E5.1} 
        \begin{table}[H]
            \centering
            \begin{tabular}{|c|c|}
                \hline I[A] & F[N] \\\hline
                \csvreader[
                late after line= \\\hline
                ]{Data/E5.1_5V_wyniki.csv}{1=\I,2=\F}
                {\I & \F}
            \end{tabular}
            \caption{Wyniki dla 5 V}
            \label{tab:wyniki dla 5 V}
        \end{table}
        \begin{table}[H]
            \centering
            \begin{tabular}{|c|c|}
                \hline I[A] & F[N] \\\hline
                \csvreader[
                late after line= \\\hline
                ]{Data/E5.1_10V_wyniki.csv}{1=\I,2=\F}
                {\I & \F}
            \end{tabular}
            \caption{Wyniki dla 10 V}
            \label{tab:wyniki dla 10 V}
        \end{table}
    \subsection{E5.2}
        Dla I = 5 V \[ B = (0.01145 \pm 0.00240) \: T \]
        Dla I = 10 V \[ B = (0.01091 \pm 0.00240) \: T \]
    \subsection{E5.3}
        \begin{table}[H]
            \centering
            \begin{tabular}{|c|c|c|}
                \hline $I_m$[A] & F[N] & B[T]\\\hline
                \csvreader[
                late after line= \\\hline
                ]{Data/E5.3_wyniki.csv}{1=\Im,2=\F,3=\B}
                {\Im & \F & \B}
            \end{tabular}
            \caption{Wymagane zależności}
            \label{tab:wyniki do E5.3}
        \end{table}
\section{Źródła}
    Sprawozdanie pisane na platformie Overleaf. \\
    Grafy rysowane przy użyciu gnuplot. \\
    Regresja liniowa policzona za pomocą komendy fit interpretera gnuplot, która w łatwy sposób pozwala na dopasowanie wybranej funkcji do posiadanych danych. \\
    Informacje brane z uzupełnienia do zadań. Wzory nieznajdujące się w instrukcji do ćwiczenia brane z Wikipedii. \\
    Cały kod źródłowy znajduje się na: \\github.com/GramNaTosterze/PG\_FIZ\_Lab3\_Spr \\oraz bezpośrednio na platformie overleaf pod adresem:\\ https://www.overleaf.com/read/pxprvzqxspfc
\end{document}